\documentclass[slidetop,11pt]{beamer}
%
\usepackage[T1]{fontenc}
\usepackage[frenchb]{babel}
% pour un pdf lisible à l'écran si on ne dispose pas  des fontes cmsuper ou lmodern
\usepackage{pslatex}

%------------- styles pour beamer ---------------
% choix des couleurs
\definecolor{fondtitre}{rgb}{255,255,255}  % vert fonce
\definecolor{coultitre}{rgb}{0.41,0.05,0.05}  % marron
\definecolor{fondtexte}{rgb}{255,255,255}     % ivoire

\colorlet{coultexte}{black}

% afficher l'ensemble du frame
\beamertemplatetransparentcovered
%
% pour les choix du menu
% supprimer les icones de navigation pour transparents
\setbeamertemplate{navigation symbols}{}

\setbeamercolor{structure}{fg=coultitre, bg=fondtitre!40}
\setbeamercolor{block body}{bg=fondtexte}
\setbeamercolor{normal text}{fg=coultexte,bg=fondtexte}

% pour le style et couleurs
\usecolortheme[RGB={15,90,150}]{structure}
\usetheme[height=7mm]{Berkeley}


% boite en couleur avec titre plus fonce
\beamerboxesdeclarecolorscheme{fondtitre}{fondtitre!70}{coultitre!20}
\beamerboxesdeclarecolorscheme{coultitre}{coultitre!70}{fondtitre!20}
%------------ fin style beamer -------------------
%
% contenu de la page de titre
\title{Solution générique de calcul GRID exploitant des messageries instantanées}
\subtitle{TER}
\author{Joffrey Hérard}
\date{\oldstylenums{28 Mars 2017}}
%
%image logo
\logo{\includegraphics[height=0.5cm]{image_logo.jpg}}

\begin{document}
% image de fond

%----------------- page de titre ----------
\frame{\titlepage}
%
%------------------ Sommaire ---------------
\section*{Sommaire}
  \frame{\frametitle{Sommaire} \small \tableofcontents}
%

\section[Introduction]{Introduction}
\setbeamercolor{frametitle}{fg=fondtitre!70}
\begin{frame}[label=debut]
\frametitle{Introduction}
\setbeamertemplate{itemize item}[raisedsquare]
\begin{itemize}
\item Calcul réparti entre plusieurs machines.
\item Évaluation de possibilité d'exécutions ou non par la machine cible.
\item C'est quoi un calcul ? Calculatoire pur ? Calcul d'images ?
\item Quelles machines sont supportées ? 
\item Quels échanges ? Quel(s) protocole(s)?
\end{itemize}
\end{frame}

\section[Modélisation]{Modélisation}
\subsection[Les Acteurs]{Les Acteurs}
\setbeamercolor{frametitle}{fg=fondtitre!70}
\begin{frame}[label=acteurs,fragile]
\frametitle{Modélisation : Les Acteurs}

\setbeamertemplate{itemize item}[raisedsquare]
\begin{itemize}
\item Fournisseur de travail/Provider, unique pour chaque travail à l'instant t.
\item Des travailleurs/Workers, de 1 à n, n définit éventuellement par le problème.
\item Un serveur qui sert de support à la messagerie instantanée.	

\end{itemize}

\begin{exampleblock}{Note :}
 
Chaque Provider, et chaque Workers, sont possiblement exécutés sur n'importe quel système d'exploitation. 
 \end{exampleblock}
\end{frame}

\subsection[Les Échanges]{Les Échanges}
\setbeamercolor{frametitle}{fg=fondtitre!70}
\begin{frame}[label=echanges]
\frametitle{Modélisation : Les Échanges}

\setbeamertemplate{itemize item}[raisedsquare]
\begin{itemize}
      \item Provider $\rightarrow$ Serveur de Messagerie
      \item Serveur de Messagerie $\rightarrow$ Worker
      \item Serveur de Messagerie $\rightarrow$ Provider
      \item Worker $\rightarrow$ Serveur de Messagerie
\end{itemize}

\end{frame}
\subsection[Représentation des JOBS]{Représentation des JOBS}
\setbeamercolor{frametitle}{fg=fondtitre!70}
\begin{frame}[label=rep_job,fragile]
\frametitle{Modélisation : Représentation des JOBS}
\setbeamertemplate{itemize item}[raisedsquare]
 
\begin{enumerate} \item ENVOI JOB :
\begin{itemize}
\item L'identifiant du problème.
\item Le code des contraintes.
\item Le code à exécuter.
\item La ligne de  commande pour l\textquoteright exécuter.
\item Le nom du fichier à exécuter.
\end{itemize}
\item REPONSE JOB : 
\begin{itemize}
\item L'identifiant pour savoir si le code a pu être exécuté.
\item L'identifiant du problème.
\item La valeur du retour de l\textquoteright exécution.
\item Code de contraintes, si on a pas pu exécuter.
\item Code exécutable, si on a pas pu exécuter.
\item Ligne de commande associée,  si on a pas pu exécuter.
\item Le nom du fichier à exécuter.
\end{itemize}
\end{enumerate}

\end{frame}

\section[Représentation des différents fichiers]{Représentation des différents fichiers}
\setbeamercolor{frametitle}{fg=fondtitre!70}
\begin{frame}[label=diff_fichiers,fragile]
\frametitle{Pourquoi cette représentation ?}
\begin{enumerate}
\item Besoin de séparations des travaux, un worker peut ne pas faire exactement la même chose que d'autre
\item Problème de chemin/d'accès sur un worker cible 
\item Perl = choix de langage parfait pour du scripting \newline $\rightarrow$ Contraintes, construction du résultat
\end{enumerate}

\end{frame}

\subsection[Fichiers de commandes]{Fichiers de commandes}
\setbeamercolor{frametitle}{fg=fondtitre!70}
\begin{frame}[label=commandes,fragile]
\frametitle{Les différents fichiers : Fichiers de commandes}
Fichier nQuenn14.dc dans le dossier Echantillon\_Script\_Cmd 
\begin{verbatim}
python @nQueen4.py 1,
python @nQueen4.py 01,
python @nQueen4.py 001,
python @nQueen4.py 0001,
\end{verbatim}
\begin{block}{Sortie regex :}
\begin{verbatim}
python JOB_REC/DATA_EXTRACT@jherard/nQueen4.py 1
python JOB_REC/DATA_EXTRACT@jherard/nQueen4.py 01
python JOB_REC/DATA_EXTRACT@jherard/nQueen4.py 001
python JOB_REC/DATA_EXTRACT@jherard/nQueen4.py 0001
\end{verbatim}
\end{block}
\end{frame}

\subsection[Fichiers de contraintes]{Fichiers de contraintes}
\setbeamercolor{frametitle}{fg=fondtitre!70}
\begin{frame}[label=contraintes,fragile]
\frametitle{Les différents fichiers : Fichiers de contraintes}

Fichier nqueen.pl dans le dossier Echantillon\_Script\_Perl 
\begin{verbatim}
#!/usr/bin/perl
use v5.14;

my $value= system("python -V");
if( $value eq 0){
    exit(3);
}
else{
    exit(0);
}
\end{verbatim}
\end{frame}

\subsection[Fichiers de Build]{Fichiers de Build}
\setbeamercolor{frametitle}{fg=fondtitre!70}
\begin{frame}[label=builds,fragile]
\frametitle{Les différents fichiers :Fichiers de Build}

Fichier build.pl dans le dossier Echantillon\_Script\_build 
\begin{verbatim}
#!/usr/bin/perl
$numArgs = $#ARGV + 1;
my $sum;
foreach $argnum (0 .. $#ARGV) {
  $sum=$sum+ $ARGV[$argnum];
}
open (FICHIER, ">resultatF.txt") || die ("Vous ne pouvez 
pas créer le fichier \"resultatF.txt\"");
print FICHIER  $sum ;
exit 0;
\end{verbatim}

\end{frame}


\subsection[Fichier d'exécution]{Fichier d'exécution}
\setbeamercolor{frametitle}{fg=fondtitre!70}
\begin{frame}[label=exec,fragile]
\frametitle{Les différents fichiers : Fichier d'exécution}

\begin{itemize}
\item Possibilité d'un code en brut
\item Possibilité d'un code qui va lui même chercher un script sur le réseau 
\item Peut être un code brut à compiler (C) ou interpréter (Python)
\end{itemize}
\end{frame}


\setbeamertemplate{itemize item}[triangle]
\section[Gestions des Erreurs]{Gestions des Erreurs}
\setbeamercolor{frametitle}{fg=fondtitre!70}
\begin{frame}[label=Gestions des Erreurs]
\frametitle{Gestions des Erreurs}
\begin{enumerate}
\item Gestions des erreurs sur l\textquoteright exécution
	\begin{itemize}
	\item Chatroom déjà existante $\rightarrow$ Message d'erreur simple.
	\item Aucun Worker ne peut exécuter le code $\rightarrow$ Mise en place d'un tableau de vérification et d'un échange supplémentaire. 
	\end{itemize}
\item Gestions des erreurs sur le réseau
	\begin{itemize}
	\item Problème de latence $\rightarrow$ essai jusqu’à 5 fois.
	\item Un Worker est déconnecté en plein calcul $ \rightarrow $ Le détecteur de présence permit par le protocole XMPP sur une Chatroom MultiUser
	\item Un Provider est déconnecté durant l'attente d'une réponse \newline $ \rightarrow $ Évaluation de présence d'un Provider, arrêter le Job et mise en place d'un heartbeat.
	
	\end{itemize}
\end{enumerate}

\end{frame}

\section[Conclusion]{Conclusion}
\setbeamercolor{frametitle}{fg=fondtitre!70}
\begin{frame}[label=conclusion]
\frametitle{Conclusion}
\begin{enumerate}
\item Les problèmes qui restent à gérer
\item OpenFire $\Rightarrow$ Montée en charge etc..
\item JOB à entrevoir $\Rightarrow$ Une réelle généricité du calcul $\Rightarrow$ Virtualisation des services
\end{enumerate}
\end{frame}

\end{document}
